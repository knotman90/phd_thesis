\chapter{Complex cellular automata}

\section{Introduction}
As stated in the section \ref{cellularAutomataIntroduction} cellular automata
are a very powerful tool for modelling complex-systems and they were adopted as
modelling and simulation paradigm in a wide range of applications like
fluid-dynamical phenomena, gas models \cite{Frisch1986}, or lattice Boltzmann
model \cite{Chopard1999}.
The behavior of those physical systems is well described  by the basic laws
of continuum mechanics (Navier-Stokes for example for fluid-dynamics),
but in some cases, when  they cannot be applied directly without adding
phenomenological assumptions or when the set of differential equation
describing the problem is not amenable to any analytical solution (except for
relaxed or particular instances) numerical solution are required.

Hence, general cases require approximated numerical methods commonly based on
space-time discretization and permitted to enlarge the set of cases which can be
carefully simulated.
But many others problems are still intractable and for those problems is
necessary to adopt new solutions. Numerical methods have became more
popular, as computational power raised up in years, and approaches in order to
overcome the problems regarding differential equations\cite{Toffoli1984} were
studied, but at the same time new methods that exploited principles of parallel
computing were adopted either for reasons of absolute performance or reasons of
cost/performance.

\section{Complex phenomena modellation with cellular automata}
Complex phenomena are different from models like Boltzmann lattice model
because of the lager scale of the space that they take into account, like lava
flow simulation models that evolve on a topographical mountain
region map\footnote{Topographical maps are usually discrete altitude value grid
of square cell not less distant than one or two meters from each other. They are
also called DEM, acronym for Digital Elevation Model}.
CA, as defined in the previous chapter are already suitable for this kind of modeling
but, an extension  was introduced by \cite{Gregorio1999} to better fit the
problem of complex phenomena modellation.


\section{Complex Cellular automata (CCA)}
While classical CA are based upon elementary automata, with few states and a
simple transition function, in order to deal with Complex phenomena it is
often necessary to allow a large number of different states a more complicated
transition. The notion of substate is introduced in the Complex case for
decomposing the state of the cell. The values associated to substates can change
in time either due to interactions among substates inside the cell (internal
transformations) or to local interactions among neighbouring cells.
When considering Complex phenomena, each cell usually maps and correspond to
a portion of space and hence a parameter that define the \emph{size }of the
space that each cell represent is needed and it's natural assume that
the cellular space is never higher than 3 (\begin{math}d \leq 3\end{math}).

\subsection{Parameters}
As said before a spatial mapping between cell size\footnote{In terms of cell
side or apothem in case respectively of square or hexagonal cell grid.} and
portion of space in which the automaton evolves is needed and it is also
reasonable to define a \emph{clock} parameter in order to map each CA step to an
effective amount of time. The choice of \emph{cell size} and \emph{clock} are
related to a particular space-time scale in which the CA model simulate the
phenomena and for this reason it has deeply consequence on the form of the model.
They are called parameters because they are predetermined and constant during
the evolution even if in some case a precise estimation mat be given only
\textit{a posteriori}. Moreover the ``best'' value for a parameter can depends 
on the value of the other parameters and modification in one of them  can have
deep consequences on the behavior of the entire model and may require the
repetition of a parameters set rearranging and recalibration phase.


\subsection{Substates}
The state of the cell has in some ways to reflect all the characteristics that
are relevant for the system's evolution. Each characteristic is mapped
to a substate which admitted values are a finite set\footnote{Continuous
quantities might be approximated by discretizing them. Note that on a computer
that is not a problem because floating point computer representations are
already discretized}.
The set of possible state of the cell is given by the cartesian product of  the
sets of substates, so, 
\[
Q = Q_1 \times Q_2 \times Q_3 \times \ldots \times Q_n
\]
When attempting in modelling complex phenomena with many relevant character-
ics could useful to create an higher hierarchy of substates and so further dividing
substates in sub-substates and taking in mind that the values of the substates are
constants within the space occupied by a cell.

\subsection{Elementary processes}\label{elementaryProcesses}
The transition function has to take into account all the possible processes that
can happen in  Complex phenomena, as physics, chemical etc. So as the state
set was decomposed in substates also the transition function is divided in so
called elementary processes that when applied in concert make up 
transition function that describe the entire process.
Moreover the elementary processes can be divided in :
\begin{description}
  \item[Internal transformations] \hfill \\
  Internal transformation \(T_1,T_2,\ldots,T_l\) define the changes in the
  values of the substates only due to interactions among them (e.g., temperature
  drop in SCIARA-fv3, see section \ref{sect:temperatureDrop}, is function of
  the substate lava thickness) inside the cell or due simply to the elapsing of
  the time (e.g., water loss by evaporation). All cell processes evolution that
  are not due to interaction with other cells can be ``labeled'' as
  \emph{internal transformation}. That is, internal transformation are those
  which would take place if all the cells were independent of each other.

For each internal transformation :
\[ 
  T_i \equiv \sigma_{t_i} \colon  S_{T_{i1}}\rightarrow S_{T_{i2}} \:,\: 1 \leq
  i \leq l
\] 
  where \begin{math}S_{T_{i1}},S_{T_{i2}} \in \mathcal{P}(Q) \end{math}(power
  set of Q) 
  \item[Local/neighborhood interaction] \hfill \\
  Local interaction \(I_1,I_2,\ldots,I_k\) describe transformation due to local
  interaction with the neighborhood cells and are often described in terms of
  flows of some quantities across them.
\[
  I_j \equiv \sigma_{I_j} \colon  S^m_{I_{j1}} \rightarrow S_{I_{j2}} \:,\:
  1 \leq j \leq k
\] 
  where \begin{math}S_{I_{j1}},S_{I_{j2}} \in \mathcal{P}(Q) \end{math}(power
  set of Q) and \(m\) is the number of the neighborhood cells.
\end{description}

The whole phenomenon can be so described by sequentially calculating internal
and local interaction functions. The execution order may be
particularly  for the model evolution and for the results\cite{Ruxton1996}.

\subsubsection{External influences}
In some scenario could it be necessary to introduce some extern influence
from the ``outside" of the automaton point of view. Those kind of influences
cannot be described in terms of local rules and typically require dedicated
functions or procedures to be defined. An example of these kind of influences
could be the lava emission from a vent of a volcano (see section \ref{mcaFormal}
for a formal definition).

\section{CCA - a formal definition}\label{mcaFormal}
Formally a Complex cellular automata CCA A is :
\[
A=<\mathbb{Z}^d,Q,P,X,\sigma,E,\gamma>
\]
where:
\begin{itemize}
  \item \begin{math}\mathbb{Z}^d=\{i=(i_1,i_1,\ldots,i_d)\mid i_k \in
  \mathbb{Z}, \forall k=1,2,\ldots,d \}\end{math} is the set of cells of the d-dimensional
   Euclidean space.
   \item \(Q=Q_1 \times Q_2 \times \ldots \times Q_n \) is the finite set of the
   finite state automaton and is the cartesian product of all the substates
   \(Q_1, Q_2, \ldots ,Q_n \).
   \item \( P=p_1, p_2, \ldots ,p_l \) is the finite set of the
   parameters
  
   \item \begin{math}X\end{math} is the neighborhood, or neighborhood template; a
  set of m \(d\)-dimensional
  \[\xi_j=\{\xi_{j1},\xi_{j2},\ldots,\xi_{jd}\} \;,\: 1\leq j \leq m\] that
  defines the set of the neighbors cells of a generic cell
  \begin{math}i=(i_1,i_1,\ldots,i_d)\end{math}
  \[
  N(X,i)=\{i+\xi_0,i+\xi_2,\ldots,i+\xi_d\}
  \] where \begin{math}\xi_0\end{math} is the null vector.
  
  \item \begin{math}\sigma=Q^m \rightarrow Q \end{math} is the transition
  function of the cells's finite state automaton . It is divided as specified
  before (see section \ref{elementaryProcesses}) in internal
  transformation,\(\sigma_{T_{1}},\sigma_{T_{1}},\ldots,\sigma_{T_{p}}\), and
  local interactions, \(\sigma_{I_{1}},\sigma_{I_{1}},\ldots,\sigma_{I_{o}}\).
  For each local interaction is possible to adopt a particular neighborhood
  template \(X_{I_k} \;,\; 1 \leq k \leq o \) and the general neighborhood
  template would be : 
\[
   D=\bigcup_{i=1}^{o}\{X_{I_k}\}
\]

\item \(E=\bigcup_{i=1}^{s}\{E_i\} \) is the set of the cells affected by
external influences.

\item \(\gamma=\{\gamma_1,\gamma_2,\ldots,\gamma_w\}\) is the set of the
functions that define external influences.
\[
\gamma_i= \mathbb{N} \times E_i \times Q \rightarrow Q \;,\; 1 \leq i \leq w
\]
where \(\mathbb{N}\) is the set of natural number, representing the current
step of the CCA.
  
  
\end{itemize}

