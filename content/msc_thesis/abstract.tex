% $Log: abstract.tex,v $ Revision 1.1  93/05/14  14:56:25  starflt Initial
% revision  Revision 1.1  90/05/04  10:41:01  lwvanels Initial revision   % The
% text of your abstract and nothing else (other than comments) goes here.
% % It will be single-spaced and the rest of the text that is supposed to go on
% % the abstract page will be generated by the abstractpage environment.  This %
% file should be \input (not \include 'd) from cover.tex.
\hfill \\
\textit{\textbf{English}}   \hfill \\
\hfill \\
 In
this thesis, a parallel version of the model SCIARA-fv3\cite{Spataro2010} was
designed and implemented using General-Purpose Computation with Graphics
Processing Units (GPGPU) and specifically, by adopting the NVIDIA Compute
Unified Device Architecture (CUDA)\cite{NvidiaprogGuide} framework in order to
improve the overall execution time. It involves the design and the application
of strategies that allow to avoid incorrect computation results due to race
conditions of any type and at the same time to achieve the best performance and
occupancy of the underlying available hardware. Carried out experiments show
that significant performance improvement in terms of speedup are achieved also
thanks to some original optimizations strategies adopted, confirming the
validity of graphics hardware as an alternative to much more expensive solutions
for the simulation of cellular automata models.
\hfill \\
\begin{center}
\line(1,0){350} \hfill \\
\end{center}
\hfill \\
\hfill \\
\textit{\textbf{Italian}}   \hfill \\
\hfill \\


In questo lavoro di tesi ho progettato ed implementato una versione parallela
del modello numero SCIARA-fv3\cite{Spataro2010} utilizzando le schede grafiche
per il calcolo general-purpose (General Purpose Computation with Graphics
Processing Units - GPGPU), adottando il  Compute Unified Device Architecture (CUDA)\cite{NvidiaprogGuide} framework
di NVIDIA con lo scopo di migliorare i tempi di esecuzione complessivi.
Questo ha comportato il design prima, e l'applicazione vera e propria poi, di
strategie che permettessero sia di evitare errori dovuti a race-conditions di
qualsiasi tipo che di raggiungere gli speedups migliori con l'hardware a
disposizione. Gli esperimenti effettuati mostrano significativi miglioramenti
nelle performance in termini di speedup grazie anche all'utilizzo di alcune stragie d'ottimizzazione nuove,
confermando la validit\`a dell'uso di processori grafici come alternativa a
soluzioni hardware per la parallelizzazione di modelli ad
automi cellulari molto pi\`u costose.



